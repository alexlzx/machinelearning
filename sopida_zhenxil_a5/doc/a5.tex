\documentclass{article}

\usepackage{fullpage}
\usepackage{color}
\usepackage{amsmath}
\usepackage{url}
\usepackage{verbatim}
\usepackage{graphicx}
\usepackage{parskip}
\usepackage{amssymb}
\usepackage{listings} % For displaying code

\begin{document}

% Colors
\definecolor{blu}{rgb}{0,0,1}
\def\blu#1{{\color{blu}#1}}
\definecolor{gre}{rgb}{0,.5,0}
\def\gre#1{{\color{gre}#1}}
\definecolor{red}{rgb}{1,0,0}
\def\red#1{{\color{red}#1}}
\def\norm#1{\|#1\|}
\newcommand{\argmin}[1]{\mathop{\hbox{argmin}}_{#1}}
\newcommand{\argmax}[1]{\mathop{\hbox{argmax}}_{#1}}
\def\R{\mathbb{R}}
\newcommand{\fig}[2]{\includegraphics[width=#1\textwidth]{#2}}
\newcommand{\centerfig}[2]{\begin{center}\includegraphics[width=#1\textwidth]{#2}\end{center}}
\def\items#1{\begin{itemize}#1\end{itemize}}
\def\enum#1{\begin{enumerate}#1\end{enumerate}}
\newcommand{\code}[1]{\lstinputlisting[language=Matlab]{#1}}
\newcommand{\half}{\frac 1 2}
\def\argmax{\mathop{\rm arg\,max}}
\def\argmin{\mathop{\rm arg\,min}}
\def\rubric#1{\gre{Rubric: \{#1\}}}{}


\title{CPSC 340 Assignment 5 (due Friday June 23 at 11:59pm)}
\date{}
\maketitle

\vspace{-7em}
\section*{Instructions}
\rubric{mechanics:3}

The above points are allocated for following the general homework instructions on GitHub.

\vspace{1em}

Note: This assignment is more open-ended than usual. There aren't always ``correct answers''. 
Try to have fun with it and learn something along the way. 
We'll mark generously... no need to go overboard!



\section{Neural Networks}
\rubric{reasoning:5}

If you run \texttt{python main.py -q 1} it will train a neural network on the \emph{basisData.pkl} data from Assignment~3. 
However, in its current form it uses no hidden layers and therefore is just performing linear regression.
In fact, if you take a look at the figure that is generated, you should see that it matches the figure
from Assignment~3 Question 1.1. 
Try to improve the performance of the neural network by adding hidden layers and tuning the other hyperparemeters. 
\blu{Hand in your plot after changing the code to have better performance, and list the changes you made. Write a couple sentences
explaining why you think your changes improved the performance. When appropriate, refer to concepts from the course like overfitting
or optimization.}

To see the list of hyperparameters and their definitions, see the scikit-learn MLPRegressor documentation:
\url{http://scikit-learn.org/stable/modules/generated/sklearn.neural_network.MLPRegressor.html}. Note: ``MLP'' stands for Multi-Layer Perceptron, 
which is another name for artificial neural network.\\
Also: we didn't discuss L-BFGS in class, but it may help to set \verb|solver| to \verb|lbfgs| since the data set is tiny.




\section{Neural Networks with Your Own Data}
\rubric{reasoning:5}

Try a neural network on the data set of your choosing. 
You can use a data set that's built into scikit-learn
(\url{http://scikit-learn.org/stable/datasets/}), or you can use a data set from a previous homework assignment, or something from an area that interests you. 
If you are doing classification rather than regression, use scikit-learn's MLPClassifier rather than the MLPRegressor that we've provided in the assignment code: \url{http://scikit-learn.org/stable/modules/generated/sklearn.neural_network.MLPClassifier.html}


\blu{Try fiddling with the hyperparameters. Write a few sentences about what you tried, what train/validation error you achieved,
and whether you think neural networks are a good model for your dataset.} Optionally (for glory rather than marks), install Keras/TensorFlow and compare it with
the scikit-learn neural net. 


\section{Recommender Systems}

This question is provided to you in the form of a Jupyter notebook, namely the file 
\emph{recommender.ipynb} in you \emph{code} directory. 
It's highly recommended for you to do 
question 3.1 directly inside the notebook, which will require
you to install Jupyter notebook and get it running. If you have Anaconda then
it will already be installed and you can fire it up by typing 
\verb|jupyter notebook recommender.ipynb|
from your terminal in the current directory. You can then run a cell with Shift+Enter. 

Note: if, upon starting the notebook, you get an ``Kernel not found'' error,
that is not a problem. Just select the version of Python you want to use from
the options provided.

If you don't want to deal with Jupyter notebook, I've also exported the notebook
to a Python a file
\emph{recommender.py}. You can work directly from there.
If you do this, you should at least read through the rendered notebook,
which you can do by navigating to it on github.ubc.ca from within your web browser. 

Whichever route you go (notebook or .py file), please make it clear in the README
and be sure to link to the appropriate file so the grader can easily find your work.

By the way, the notebook file is quite long and so it may look very intimidating.
However, most of it is code that's already provided for you, which you just need to read and interpret.
The amount of code you need to write isn't much at all.

\subsection{Implementation}
\rubric{code:8}

Read through the notebook file up to the section entitled \emph{The experiments: methods 1-9}. 
We've implemented the following models for you already: (1), (2), (3), (5), (7), (8), most of (9). 
\blu{Implement the remaining cases (4), (6), and the rest of (9).} Some preprocessing is already done for you in (4).

\subsection{Comparing methods}
\rubric{reasoning:5}

If you are using the Jupyter Notebook, the results are conveniently summarized for you 
in a table at the bottom of the file.
\blu{Write a few sentences discussing which methods did better, and why.}
Discuss this in the context of the fundamental tradeoff: did some methods over/underfit?


\subsection{Ridge regression}
\rubric{reasoning:2}

As mentioned briefly in class, ``Ridge regression'' refers to linear regression with L2-regularization. 
\blu{Explain why it makes sense to use ridge regression in (6) but not really for (5).}
Include both a theoretical/intuitive explanation and a quick empirical assessment.

\subsection{Stochastic gradient}
\rubric{reasoning:2}

\blu{Under what circumstances would it make sense to use stochastic gradient descent instead
of gradient descent for the models (8) and (9)? What are the pros/cons of doing so?}

\subsection{Validation}
\rubric{reasoning:2}

We divided the data into training/validation sets by splitting the ratings.
However, you might also do this by splitting the data differently. For example,
you could split by movies or by users (e.g., the first 5000 movies are used for training, the rest for validation). 
\blu{What effect do you anticipate the splitting method
would have on your different models?}


\subsection{Next steps}
\rubric{reasoning:5}

Personally, I find the results of even model (9) to be disappointing.
\blu{Discuss some ideas for how you might do better.} Some ideas to get you thinking:
\begin{enumerate}
\item What other models might you try? 
\item Do we need regularization? If so, what type(s)?
\item Is the squared error loss satisfactory?
\item Are we solving the optimization problem effectively?
\item What new data might you collect? 
\item Is our train/validation procedure reasonable?
\end{enumerate}

\end{document}