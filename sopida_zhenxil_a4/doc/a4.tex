\documentclass{article}


\usepackage{fullpage}
\usepackage{color}
\usepackage{amsmath}
\usepackage{url}
\usepackage{verbatim}
\usepackage{graphicx}
\usepackage{parskip}
\usepackage{amssymb}
\usepackage{listings} % For displaying code

\begin{document}

% Colors
\definecolor{blu}{rgb}{0,0,1}
\def\blu#1{{\color{blu}#1}}
\definecolor{gre}{rgb}{0,.5,0}
\def\gre#1{{\color{gre}#1}}
\definecolor{red}{rgb}{1,0,0}
\def\red#1{{\color{red}#1}}
\def\norm#1{\|#1\|}
\newcommand{\argmin}[1]{\mathop{\hbox{argmin}}_{#1}}
\newcommand{\argmax}[1]{\mathop{\hbox{argmax}}_{#1}}
\def\R{\mathbb{R}}
\newcommand{\fig}[2]{\includegraphics[width=#1\textwidth]{#2}}
\newcommand{\centerfig}[2]{\begin{center}\includegraphics[width=#1\textwidth]{#2}\end{center}}
\def\items#1{\begin{itemize}#1\end{itemize}}
\def\enum#1{\begin{enumerate}#1\end{enumerate}}
\newcommand{\half}{\frac 1 2}
\def\argmax{\mathop{\rm arg\,max}}
\def\argmin{\mathop{\rm arg\,min}}
\def\rubric#1{\gre{Rubric: \{#1\}}}{}


\title{CPSC 340 Assignment 4 (due Friday June 16 at 11:59pm)}
\date{}
\maketitle

\vspace{-7em}

\section*{Instructions}
\rubric{mechanics:3}

The above points are allocated for following the general homework instructions.

\vspace{1em}

\textbf{A note on Python 2}: if you are using Python 2.7, please add
\begin{verbatim}
from __future__ import division
\end{verbatim}
to the top of each Python file. 

\textbf{**Regardless of what version of Python you are using**}:
You'll need to grab data files from the ``home'' repo on GitHub for this assignment.


\section{MAP Estimation}
\rubric{reasoning:5}

In class, we considered MAP estimation in a regression model where we assumed that:
\items{
\item The likelihood $p(y_i | x_i, w)$ is a normal distribution with a mean of $w^Tx_i$ and a variance of $1$.
\item The prior for each variable $j$, $p(w_j)$, is a normal distribution with a mean of zero and a variance of $\lambda^{-1}$.
}
Under these assumptions, we showed that this leads to the standard L2-regularized least squares objective function:
\[
f(w) = \frac{1}{2}\norm{Xw - y}^2 + \frac \lambda 2 \norm{w}^2.
\]
\blu{For each of the alternate assumptions below, show how the loss function would change} (simplifying as much as possible):
\enum{
\item We use a zero-mean Laplace prior for each variable with a scale parameter of $\lambda^{-1}$, so that 
\[
p(w_j) = \frac{\lambda}{2}\exp(-\lambda|w_j|).
\]
\item We use a Laplace likelihood with a mean of $w^Tx_i$ and a scale of $1$, so that 
\[
p(y_i | x_i, w) = \frac 1 2 \exp(-|w^Tx_i - y_i|).
\]
\item We use a Gaussian likelihood where each datapoint where the variance is $\sigma^2$ instead of $1$,
\[
p(y_i | x_i,w) = \frac{1}{\sqrt{2\sigma^2\pi}}\exp\left(-\frac{(w^Tx_i - y_i)^2}{2\sigma^2}\right).
\]
\item We use a Gaussian likelihood where each datapoint has its own variance $\sigma_i^2$,
\[
p(y_i | x_i,w) = \frac{1}{\sqrt{2\sigma_i^2\pi}}\exp\left(-\frac{(w^Tx_i - y_i)^2}{2\sigma_i^2}\right).
\]
\item We use a (very robust) student $t$ likelihood with a mean of $w^Tx_i$ and a degree of freedom of $\nu$,
\[
p(y_i | x_i, w) = \frac{\Gamma\left(\frac{\nu + 1}{2}\right)}{\sqrt{\nu\pi}\Gamma\left(\frac \nu 2\right)}\left(1 + \frac{(w^Tx_i - y_i)^2}{\nu}\right)^{-\frac{\nu+1}{2}},
\] 
where $\Gamma$ is the ``gamma" function (which is always non-negative).
}
\blu{Why is loss coming from the student $t$ distribution ``very robust"?}





\section{Naive Bayes}

In this section we'll implement naive Bayes, 
a very fast classification method that is often surprisingly accurate for text data with simple representations like bag of words.



\subsection{Naive Bayes by Hand}
\rubric{reasoning:3}
Consider the dataset below, which has $10$ training examples and $2$ features:
\[
X = \begin{bmatrix}0 & 1\\1 & 1\\ 0 & 0\\ 1 & 1\\ 1 & 1\\ 0 & 0\\  1 & 0\\  1 & 0\\  1 & 1\\  1 &0\end{bmatrix}, \quad y = \begin{bmatrix}1\\1\\1\\1\\1\\1\\0\\0\\0\\0\end{bmatrix}.
\]
Suppose you believe that a naive Bayes model would be appropriate for this dataset, and you want to classify the following test example:
\[
\hat{x} = \begin{bmatrix}1 & 1\end{bmatrix}.
\]

\blu{(a) Compute the estimates of the class prior probabilities:}
\items{
\item$ p(y = 1)$.
\item $p(y = 0)$.
}

\blu{(b) Compute the estimates of the 4 conditional probabilities required by naive Bayes for this example:}
\items{
\item $p(x_1 = 1 | y = 1)$.
\item $p(x_2 = 1 | y = 1)$.
\item $p(x_1 = 1 | y = 0)$.
\item $p(x_2 = 1 | y = 0)$.
}

\blu{(c) Under the naive Bayes model and your estimates of the above probabilities, what is the most likely label for the test example? (Show your work.)}




\subsection{Naive Bayes Implementation}
\rubric{code:3}

If you run \texttt{python main.py \string-q 2.2} it will first load the following dataset:
\enum{
\item $groupnames$: The names of four newsgroups.
\item $wordlist$: A list of words that occur in posts to these newsgroups.
\item $X$: A sparse binary matrix. Each row corresponds to a post, and each column corresponds to a word from the word list. A value of $1$ means that the word occured in the post.
\item $y$: A vector with values $0$ through $3$, with the value corresponding to the newsgroup that the post came from.
\item $Xvalidate$ and $yvalidate$: the word lists and newsgroup labels for additional newsgroup posts.
}
It will train a random forest and report its validation error, followed by training a naive Bayes model.

While the \emph{predict} function of the naive Bayes classifier is already implemented, the calculation of the variable $p\_xy$ in \emph{fit} is incorrect 
(right now, it just sets all values to $1/2$). \blu{Modify this function so that \emph{p\_xy} correctly computes the conditional probability 
of these values based on the frequencies in the data set. Hand in your code and report the validation error that you obtain.}



\subsection{Runtime of Naive Bayes for Discrete Data}
\rubric{reasoning:2}

Assume you have the following setup:
\items{
\item The training set has $n$ objects each with $d$ features.
\item The test set has $t$ objects with $d$ features.
\item Each feature can have up to $c$ discrete values (you can assume $c \leq n$).
\item There are $k$ class labels (you can assume $k \leq n$)
}
You can implement the training phase of a naive Bayes classifier in this setup in $O(nd)$, s
ince you only need to do a constant amount of work for each $X(i,j)$ value. 
(You do not have to actually implement it in this way for the previous question, but you should think about how this could be done). 
\blu{What is the cost of classifying $t$ test examples with the model?}




\section{Principal Component Analysis}

\subsection{PCA by Hand}
\rubric{reasoning:3}


Consider the following dataset, containing 5 examples with 2 features each:
\begin{center}
\begin{tabular}{cc}
$x_1$ & $x_2$\\
\hline
-2 & -1\\
-1 & 0\\
0 & 1\\
1 & 2\\
2 & 3\\
\end{tabular}
\end{center}
Recall that with PCA we usually assume that the PCs are normalized ($\norm{w} = 1$), we need to center the data before we apply PCA, and that the direction of the first PC is the one that minimizes the orthogonal distance to all data points.
\blu{
\enum{
\item What is the first principal component?
\item What is the (L2-norm) reconstruction error of the point (3,3)? (Show your work.)
\item What is the (L2-norm) reconstruction error of the point (3,4)? (Show your work.)
}
}


\subsection{Data Visualization}
\rubric{reasoning:2}

The command \emph{main -q 3.2} will load an animals dataset assignment, standardize the features, and then give two unsatisfying visualizations of it. 
First it shows a plot of the matrix entries, which has too much information and thus gives little insight into the relationships between the animals. 
Next it shows a scatterplot based on two random features. 
We label some random points, but because of the binary features even a scatterplot matrix will show us almost nothing about the data.

The class \emph{pca.PCA} applies the classic PCA method (orthogonal bases via SVD) for a given $k$. 
Use this class so that the scatterplot uses the latent features $z_i$ from the PCA model. 
Make a scatterplot of the two columns in $Z$, and label a bunch of the points in the scatterplot. \blu{Hand in your code and the scatterplot}.
 

\subsection{Data Compression}
\rubric{reasoning:2}

It is important to know how much of the information in our dataset is captured by the low-dimensional PCA representation.
In class we discussed the ``analysis" view that PCA maximizes the variance that is explained by the PCs, and the connection between the Frobenius norm and the variance of a centered data matrix $X$. Use this connection to answer the following:
\blu{\enum{
\item How much of the variance is explained by our 2-dimensional representation from the previous question?
\item How many PCs are required to explain 50\% of the variance in the data?
}}


\section{Robust PCA}

\rubric{code:4}

The command \emph{main -q 4} loads a dataset $X$ where each row contains the pixels from a single frame of a video of a highway. The demo applies PCA to this dataset and then uses this to reconstruct the original image. 
It then shows the following 3 images for each frame (pausing and waiting for input between each frame):
\enum{
\item The original frame.
\item The reconstruction based on PCA.
\item A binary image showing locations where the reconstruction error is non-trivial.
}
Recently, latent-factor models have been proposed as a strategy for ``background subtraction'': trying to separate objects from their background. In this case, the background is the highway and the objects are the cars on the highway. In this demo, we see that PCA does an ok job of identifying the cars on the highway in that it does tend to identify the locations of cars. However, the results aren't great as it identifies quite a few irrelevant parts of the image as objects.

Robust PCA is a variation on PCA where we replace the L2-norm with the L1-norm,
\[
f(Z,W) = \sum_{i=1}^n\sum_{j=1}^d |w_j^Tz_i - x_{ij}|,
\]
and it has recently been proposed as a more effective model for background subtraction. \blu{Complete the class \emph{pca.RobustPCA}, 
that uses a smooth approximation to the absolute value to implement robust PCA.}

Hint: most of the work has been done for you in the class \emph{pca.AlternativePCA}. 
This work implements an alternating minimization approach to minimizing the (squared) PCA objective (without enforcing orthogonality). This gradient-based approach to PCA can be modified to use a smooth approximation of the L1-norm. Note that the log-sum-exp approximation to the absolute value may be hard to get working due to numerical issues, and a numerically-nicer approach is to use the ``multi-quadric'' approximation:
\[
|\alpha| \approx \sqrt{\alpha^2 + \epsilon},
\]
where $\epsilon$ controls the accuracy of the approximation (a typical value of $\epsilon$ is $0.0001$).





\section{Multi-Dimensional Scaling}

The command \emph{main -q 5} loads the animals dataset and then applies gradient dsecent to minimize the following multi-dimensional scaling (MDS) objective (starting from the PCA solution):
\begin{equation}
\label{eq:MDS}
f(Z) =  \frac{1}{2}\sum_{i=1}^n\sum_{j=i+1}^n (  \norm{z_i - z_j} - \norm{x_i - x_j})^2.
\end{equation}
 The result of applying MDS is shown below.
\centerfig{.5}{../figs/MDS_animals.png}
Although this visualization isn't perfect (with ``gorilla'' being placed close to the dogs and ``otter'' being placed close to two types of bears), this visualization does organize the animals in a mostly-logical way.


\subsection{ISOMAP}
\rubric{code:4}

Euclidean distances between very different animals are unlikely to be particularly meaningful. 
However, since related animals tend to share similar traits we might expect the animals to live on a low-dimensional manifold. 
This suggests that ISOMAP may give a better visualization. 
Fill in the class \emph{ISOMAP} so that it computes the approximate geodesic distance 
(shortest path through a graph where the edges are only between nodes that are $k$-nearest neighbour) between each pair of points, 
and then fits a standard MDS model~\eqref{eq:MDS} using gradient descent. \blu{Plot the results using $2$ and using $3$-nearest neighbours}.

The function \emph{utils.dijskstra} can be used to compute the shortest (weighted) distance between two points in a weighted graph. 
This function requires an $n \times n$ matrix giving the weights on each edge (use $0$ as the weight for absent edges). 
Note that ISOMAP uses an undirected graph, while the $k$-nearest neighbour graph might be asymmetric. 
One of the usual heuristics to turn this into a undirected graph is to include an edge $i$ to $j$ if $i$ is a KNN of $j$ or if $j$ is a KNN of $i$. 
(Another possibility is to include an edge only if $i$ and $j$ are mutually KNNs.)



\subsection{Reflection}
\rubric{reasoning:2}

\blu{Briefly comment on PCA (Q1.2) vs. MDS vs. ISOMAP for dimensionality reduction on this particular data set. In your opinion, which method did the best job and why?}



\end{document}